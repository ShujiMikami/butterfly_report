\documentclass{jsarticle}

\usepackage[dvipdfmx]{graphicx}

\title{キタテハ観察日誌}

\begin{document}
\maketitle

\section{6/20の記録}
\subsection{15時:キタテハの幼虫確保}
これまで, タテハチョウ科の幼虫は, ルリタテハ, アカタテハ, オオムラサキ, メスグロヒョウモンしか飼育したことがなく, 
割と近くで見ることの多い, キタテハを育てたことがなく, そもそも何が食草なのかも知らなかった. 
調べてみると, カナムグラというつる植物で, 正直どこにでもあるイメージの植物であった. 
それゆえ, 近所でどこでも見つかるのでは, と思い, 探してみると, 中山から家まで帰る裏道の一か所に生えているのを確認, 
通るたびに見ていたが, 幼虫も卵も見つからない. また, 母蝶が産卵に来たように見えたが, 何もせずに帰ってしまったので, 環境として不適切だったのだろう. 
このカナムグラであるが, 四季の森公園に, 腐るほどあったように記憶していたので, 行ってみると, やはりあった(とはいえ, 思っていた場所にはなく, 別の場所だったが).
タテハチョウは, 非常にわかりやすく葉の表に卵を産み付けるので, 割と簡単に見つかるかと思ったが, まったく見つからない. 
しかし, 葉の裏に2齢程度と思われる幼虫がすぐ見つかり, 確保. さらに, その後, 巣(とはいえ, アカタテハに比べると粗末なもの)の中をのぞくと, 見つかること3匹, 計4匹確保できた. 
そのうち2匹は, 同じ葉にいた. 
最初に見つけた幼虫を見る限り, 若いころは, あまりちゃんと巣をつくらないようだ. 

\end{document}
