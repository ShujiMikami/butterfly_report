\documentclass{jsarticle}

\usepackage[dvipdfmx]{graphicx}

\title{モンキチョウ観察日誌}

\begin{document}
\maketitle

\section{6/8の記録}
\subsection{14時:鶴見川沿い堤防}
この日は, 会社に午後から出社したが, 出社途上で, モンキチョウが, 頻繁に草に止まっては飛び立ちを繰り返す, いかにも産卵していると思われる状態を確認した. 
もともと鶴見川沿いでは, モンキチョウをよく見かけてたのと, 食草のアカツメクサが多く生えているので, 繁殖していても不思議はなかった. 
そもそも, 非常によく見る種であるが, 実は一度も, その生態を成虫以外見たことがなかった. 
これ幸いと, 蝶が止まっていた付近を見ると, やはりアカツメクサがあり, その葉をよく見ると, 1mm程度の極細の卵が葉の表面に産み付けられているのを確認した. 
その周囲の株を, ちょっと見ただけで, あちこちに卵が見つかり, 白っぽい黄色のものや, 濃い橙色のものもみつかった. 
あとで調べたところによると, 産卵直後が白く, 赤くなってついには黒くなるらしい. 
探せばいくらでも見つかりそうだったが, とりあえず7卵確保した. 
産卵場所は専ら葉であるが, 根元, 葉先, 裏表など, どこにでも適当に産み付けるようだ. 
なお, 母蝶は, 白色型であった. 

\section{6/9の記録}
\subsection{14時:卵の一つがくろくなった}
卵の一つが, 橙色を経由せずにいきなり黒くなった. 孵化の予兆か. 

\subsection{20時:卵が一つ孵化した}
黒くなったものとは違う卵が孵化していた. シロチョウ系の幼虫の例に漏れず, 卵の空を食べていた. 
黒くなったものは, どうも孵化しないかもしれない. 
なお, 孵化後, あまり移動しないようだ. 

\subsection{23時:幼虫が動き出した}
最初に孵化した幼虫(以下幼虫1)が, 孵化後3,4時間程度でうねうね動き出した. こちらには, アカツメクサを与えて育てて見ることとした. 今回, 個体数を半分に分け, 片方はアカツメクサ, もう片方はシロツメクサにし, 何か違いが出るか見てみようと思う. 


\end{document}
